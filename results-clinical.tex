\documentclass[]{article}
\usepackage{lmodern}
\usepackage{amssymb,amsmath}
\usepackage{ifxetex,ifluatex}
\usepackage{fixltx2e} % provides \textsubscript
\ifnum 0\ifxetex 1\fi\ifluatex 1\fi=0 % if pdftex
  \usepackage[T1]{fontenc}
  \usepackage[utf8]{inputenc}
\else % if luatex or xelatex
  \ifxetex
    \usepackage{mathspec}
  \else
    \usepackage{fontspec}
  \fi
  \defaultfontfeatures{Ligatures=TeX,Scale=MatchLowercase}
\fi
% use upquote if available, for straight quotes in verbatim environments
\IfFileExists{upquote.sty}{\usepackage{upquote}}{}
% use microtype if available
\IfFileExists{microtype.sty}{%
\usepackage{microtype}
\UseMicrotypeSet[protrusion]{basicmath} % disable protrusion for tt fonts
}{}
\usepackage[margin=1in]{geometry}
\usepackage{hyperref}
\hypersetup{unicode=true,
            pdftitle={Network changes after stroke -- results of longitudinal SFB data},
            pdfauthor={Eckhard Schlemm},
            pdfborder={0 0 0},
            breaklinks=true}
\urlstyle{same}  % don't use monospace font for urls
\usepackage{color}
\usepackage{fancyvrb}
\newcommand{\VerbBar}{|}
\newcommand{\VERB}{\Verb[commandchars=\\\{\}]}
\DefineVerbatimEnvironment{Highlighting}{Verbatim}{commandchars=\\\{\}}
% Add ',fontsize=\small' for more characters per line
\usepackage{framed}
\definecolor{shadecolor}{RGB}{248,248,248}
\newenvironment{Shaded}{\begin{snugshade}}{\end{snugshade}}
\newcommand{\AlertTok}[1]{\textcolor[rgb]{0.94,0.16,0.16}{#1}}
\newcommand{\AnnotationTok}[1]{\textcolor[rgb]{0.56,0.35,0.01}{\textbf{\textit{#1}}}}
\newcommand{\AttributeTok}[1]{\textcolor[rgb]{0.77,0.63,0.00}{#1}}
\newcommand{\BaseNTok}[1]{\textcolor[rgb]{0.00,0.00,0.81}{#1}}
\newcommand{\BuiltInTok}[1]{#1}
\newcommand{\CharTok}[1]{\textcolor[rgb]{0.31,0.60,0.02}{#1}}
\newcommand{\CommentTok}[1]{\textcolor[rgb]{0.56,0.35,0.01}{\textit{#1}}}
\newcommand{\CommentVarTok}[1]{\textcolor[rgb]{0.56,0.35,0.01}{\textbf{\textit{#1}}}}
\newcommand{\ConstantTok}[1]{\textcolor[rgb]{0.00,0.00,0.00}{#1}}
\newcommand{\ControlFlowTok}[1]{\textcolor[rgb]{0.13,0.29,0.53}{\textbf{#1}}}
\newcommand{\DataTypeTok}[1]{\textcolor[rgb]{0.13,0.29,0.53}{#1}}
\newcommand{\DecValTok}[1]{\textcolor[rgb]{0.00,0.00,0.81}{#1}}
\newcommand{\DocumentationTok}[1]{\textcolor[rgb]{0.56,0.35,0.01}{\textbf{\textit{#1}}}}
\newcommand{\ErrorTok}[1]{\textcolor[rgb]{0.64,0.00,0.00}{\textbf{#1}}}
\newcommand{\ExtensionTok}[1]{#1}
\newcommand{\FloatTok}[1]{\textcolor[rgb]{0.00,0.00,0.81}{#1}}
\newcommand{\FunctionTok}[1]{\textcolor[rgb]{0.00,0.00,0.00}{#1}}
\newcommand{\ImportTok}[1]{#1}
\newcommand{\InformationTok}[1]{\textcolor[rgb]{0.56,0.35,0.01}{\textbf{\textit{#1}}}}
\newcommand{\KeywordTok}[1]{\textcolor[rgb]{0.13,0.29,0.53}{\textbf{#1}}}
\newcommand{\NormalTok}[1]{#1}
\newcommand{\OperatorTok}[1]{\textcolor[rgb]{0.81,0.36,0.00}{\textbf{#1}}}
\newcommand{\OtherTok}[1]{\textcolor[rgb]{0.56,0.35,0.01}{#1}}
\newcommand{\PreprocessorTok}[1]{\textcolor[rgb]{0.56,0.35,0.01}{\textit{#1}}}
\newcommand{\RegionMarkerTok}[1]{#1}
\newcommand{\SpecialCharTok}[1]{\textcolor[rgb]{0.00,0.00,0.00}{#1}}
\newcommand{\SpecialStringTok}[1]{\textcolor[rgb]{0.31,0.60,0.02}{#1}}
\newcommand{\StringTok}[1]{\textcolor[rgb]{0.31,0.60,0.02}{#1}}
\newcommand{\VariableTok}[1]{\textcolor[rgb]{0.00,0.00,0.00}{#1}}
\newcommand{\VerbatimStringTok}[1]{\textcolor[rgb]{0.31,0.60,0.02}{#1}}
\newcommand{\WarningTok}[1]{\textcolor[rgb]{0.56,0.35,0.01}{\textbf{\textit{#1}}}}
\usepackage{graphicx,grffile}
\makeatletter
\def\maxwidth{\ifdim\Gin@nat@width>\linewidth\linewidth\else\Gin@nat@width\fi}
\def\maxheight{\ifdim\Gin@nat@height>\textheight\textheight\else\Gin@nat@height\fi}
\makeatother
% Scale images if necessary, so that they will not overflow the page
% margins by default, and it is still possible to overwrite the defaults
% using explicit options in \includegraphics[width, height, ...]{}
\setkeys{Gin}{width=\maxwidth,height=\maxheight,keepaspectratio}
\IfFileExists{parskip.sty}{%
\usepackage{parskip}
}{% else
\setlength{\parindent}{0pt}
\setlength{\parskip}{6pt plus 2pt minus 1pt}
}
\setlength{\emergencystretch}{3em}  % prevent overfull lines
\providecommand{\tightlist}{%
  \setlength{\itemsep}{0pt}\setlength{\parskip}{0pt}}
\setcounter{secnumdepth}{0}
% Redefines (sub)paragraphs to behave more like sections
\ifx\paragraph\undefined\else
\let\oldparagraph\paragraph
\renewcommand{\paragraph}[1]{\oldparagraph{#1}\mbox{}}
\fi
\ifx\subparagraph\undefined\else
\let\oldsubparagraph\subparagraph
\renewcommand{\subparagraph}[1]{\oldsubparagraph{#1}\mbox{}}
\fi

%%% Use protect on footnotes to avoid problems with footnotes in titles
\let\rmarkdownfootnote\footnote%
\def\footnote{\protect\rmarkdownfootnote}

%%% Change title format to be more compact
\usepackage{titling}

% Create subtitle command for use in maketitle
\providecommand{\subtitle}[1]{
  \posttitle{
    \begin{center}\large#1\end{center}
    }
}

\setlength{\droptitle}{-2em}

  \title{Network changes after stroke -- results of longitudinal SFB data}
    \pretitle{\vspace{\droptitle}\centering\huge}
  \posttitle{\par}
    \author{Eckhard Schlemm}
    \preauthor{\centering\large\emph}
  \postauthor{\par}
    \date{}
    \predate{}\postdate{}
  
\usepackage{booktabs}
\usepackage{longtable}
\usepackage{array}
\usepackage{multirow}
\usepackage{wrapfig}
\usepackage{float}
\usepackage{colortbl}
\usepackage{pdflscape}
\usepackage{tabu}
\usepackage{threeparttable}
\usepackage{threeparttablex}
\usepackage[normalem]{ulem}
\usepackage{makecell}
\usepackage{xcolor}

\begin{document}
\maketitle

\hypertarget{results}{%
\section{Results}\label{results}}

After initial screening of consecutive patients, a total of 59 patients
were considered for inclusion in the study. At the end of the
recruitment period, 30 patients with subcortical infarcts in the
territory of the middle cerebral artery and diffusion MRI from at least
two timepoints were available for analysis (Fig. 1). Further details of
the available data are presented in the Supplement.

Imaging and clinical testing in the acute phase took place after a
median of 4 (IQR{[}1,5{]}) days. Assessment in the subacute and chronic
phases was performed 4.92857142857143
({[}4.28571428571429,6.14285714285714{]}), 14.7142857142857
({[}13.2857142857143,16.1428571428571{]}) and 51.8571428571429
({[}50,53.4285714285714{]}) weeks after stroke, respectively.

\hypertarget{clinical-data}{%
\subsection{Clinical data}\label{clinical-data}}

\hypertarget{baseline-demographics}{%
\subsubsection{Baseline demographics}\label{baseline-demographics}}

Of the 27 stroke patients included in the study, 10 were female; their
age was 65.5926 +/- 12.1125 (mean +/- standard deviation); 18
(66.6667\%, CI\textsubscript{95} {[}46.015, 82.7647{]}\%) had a lesion
in the left hemisphere; the infarct volume as measured at the first time
point, 3-5 days after stroke, ranged from 0.61 ml to 69.15 ml (median
1.85 ml, IQR {[}1.53, 12.44{]} ml). The lesions were predominantly
located in subcortical brain areas, involving the centrum ovale, the
corona radiata and the internal capsule (Fig0). There was no
statistically significant association of lesion volume with side of the
lesion (d=-0.4849, t\textsubscript{25}=1.1879, p=0.246).

Initial severity of stroke symptoms ranged from 0 to 13 on the NIH
Stroke Scale (median 3.5, IQR {[}2.25,6.5{]}). Quasi-Poisson regressions
indicated that patients with larger infarct volumes were affected more
severely at the acute (p\textsubscript{3-5d}=0.0221), but not at the
later time points at one, three or twelve months after stroke. There was
no effect of side of the lesion, nor age or sex of the patient on stroke
severity.

Impairments in strength and dexterity of the affected hand were
quantified in the acute phase as relative grip strength ranging from 0
to 1.12 (median 0.68, IQR {[}0.34, 0.84{]}) and Fugl-Meyer score ranging
from 4 to 66 (median 56, IQR {[}27.5, 62.75{]}). In these motor specific
outcome measures there was no statistically significant association with
volume or side of the lesion, nor with age or sex of the patient.

\hypertarget{time-course-of-symptom-severity-and-motor-function}{%
\subsubsection{Time course of symptom severity and motor
function}\label{time-course-of-symptom-severity-and-motor-function}}

Over the course of the study most patients improved clinically. The
median NIHSS score, the ratio of grip strength in affected to unaffected
hand, and FM score improved to 0 (IQR {[}0, 2.5{]}), 0.96 (IQR
{[}0.8575, 1.0225{]}) and 66 (IQR {[}61.5, 66{]}) at 12 months
follow-up, respectively (Fig. 1). Growth curve analyses indicated
statistical superiority of exponential models
(AIC\textsubscript{exp}\textsuperscript{NIHSS} = 449.6305,
AIC\textsubscript{exp}\textsuperscript{GS} = -13.0516,
AIC\textsubscript{exp}\textsuperscript{FM} = 821.1731) over linear fits
(AIC\textsubscript{lin}\textsuperscript{NIHSS} = 480.7426,
AIC\textsubscript{lin}\textsuperscript{GS} = 15.2338,
AIC\textsubscript{lin}\textsuperscript{FM} = 849.85) for each of the
three outcome variables (Tab. 1).

\begin{figure}
\centering
\includegraphics{results-clinical_files/figure-latex/unnamed-chunk-10-1.pdf}
\caption{Fig. 1: Temporal profiles of clinical outcome parameters.
Horizontal axes indicate time after stroke. Thin lines represent
linearly interpolated profiles for individual patients. Circles and bars
denote cross-sectional means and asymptotic standard errors,
respectively. Thick lines visualize the non-linear model
\(\mathrm{Outcome}_t\sim a + \Delta(1-\exp(-b\,t))\). NIHSS=National
Institutes of Health Stroke Scale, d=days, m=months.}
\end{figure}

\begin{Shaded}
\begin{Highlighting}[]
\NormalTok{temp}
\end{Highlighting}
\end{Shaded}

\begin{verbatim}
## TypeError: Attempting to change the setter of an unconfigurable property.
## TypeError: Attempting to change the setter of an unconfigurable property.
\end{verbatim}

\includegraphics[width=4.08in,height=2.35in,keepaspectratio]{results-clinical_files/figure-latex/unnamed-chunk-11-1.png}
Tab. 1: Point estimates, standard errors and p-values for model
parameters \(a\) (initial value), \(b\) (rate of change), and Δ (total
amount of change) obtained from fitting the exponential model (1) to
temporal profiles of clinical outcome parameters. Standard errors and
\(p\)-values result from non-linear mixed-effects regressions fit using
the R package nlme. FM = Fugl-Meyer, NIHSS = National Institutes of
Health Stroke Scale, rGS = relative grip strength.

\hypertarget{network-properties}{%
\subsection{Network properties}\label{network-properties}}

The mean network density, i.e.~the proportion of non-zero connections,
was (95.6 +/- 1.8) \% with no significant differences between left and
right hemispheres or between time points.

\hypertarget{effects-of-time-and-lesion-status}{%
\subsubsection{Effects of time and lesion
status}\label{effects-of-time-and-lesion-status}}

\hypertarget{numerical-global-connectivity}{%
\paragraph{Numerical global
connectivity}\label{numerical-global-connectivity}}

Analysis of numerical measures of intrahemispheric connectivity revealed
that, based on the Akaike information criterion (AIC), the time course
of median edge weight was better described by an exponential than a
linear model (AIC\textsubscript{exp} = -1014.5272,
AIC\textsubscript{lin} = -967.8284). The temporal profiles of
intrahemispheric \(q_{50}\), depicted in Fig. 3, did not differ
significantly between ipsi- and contralesional hemispheres with a trend
towards larger decline in stroke hemispheres
(Δ\textsubscript{ipsi}=-9e-04 +/- 0.0035, p=0.8071). Subgroup modelling
showed a significant exponential decline of median edge weight in stroke
hemispheres (Δ=-0.0068 +/- 0.0036, p=0.0605; AIC\textsubscript{exp} =
-539.1404, AIC\textsubscript{lin} = -517.7656), but did not reveal a
significant effect of time on connectivity in contralesional hemispheres
(Δ=-0.0019 +/- 0.0045, p=0.6694; AIC\textsubscript{exp} = -486.0607,
AIC\textsubscript{lin} = -464.8828). Further details, including
estimates of the nuisance model parameters a and b are given in
supplementary table S-Tab. 2.

\hypertarget{global-network-architecture}{%
\paragraph{Global network
architecture}\label{global-network-architecture}}

Growth curve analysis of whole-brain global graph parameters using
non-linear mixed-effects regression modelling revealed consistent
effects of time (Fig. 3, Tab. 2). Global efficiency declined
exponentially over time in stroke but not intact hemispheres. Modularity
increased significantly in both stroke and intact hemispheres, with a
numerically larger effect ipsilesionally. These effects were not
sensitive to the choice of network density and persisted over a wide
range of thresholds (Supplement). Inclusion of age and sex as nuisance
regressors did not substantially change the results (not shown).
\includegraphics{results-clinical_files/figure-latex/unnamed-chunk-14-1.pdf}

\begin{Shaded}
\begin{Highlighting}[]
\NormalTok{temp}
\end{Highlighting}
\end{Shaded}

\begin{verbatim}
## TypeError: Attempting to change the setter of an unconfigurable property.
## TypeError: Attempting to change the setter of an unconfigurable property.
\end{verbatim}

\includegraphics[width=7.23in,height=2.53in,keepaspectratio]{results-clinical_files/figure-latex/unnamed-chunk-16-1.png}
Tab. 2: Point estimates, standard errors, and p-values of model
parameters obtained from fitting the exponential model (1) to temporal
profiles of intrahemispheric global graph parameters. In the joint model
, the parameter of topological change, Δ, was allowed to vary between
stroke and intact hemispheres. Standard errors and \(p\)-values result
from non-linear mixed-effects regressions fit either jointly (`joint
model') or separately for stroke and intact hemispheres, using the R
package nlme. GGP = Global graph parameter, s.e. = standard error.

\hypertarget{local-network-architecture}{%
\paragraph{Local network
architecture}\label{local-network-architecture}}

\hypertarget{association-of-network-properties-with-lesion-volume}{%
\subsubsection{Association of network properties with lesion
volume}\label{association-of-network-properties-with-lesion-volume}}

\hypertarget{numerical-connectivity-and-global-graph-parameters}{%
\paragraph{Numerical connectivity and global graph
parameters}\label{numerical-connectivity-and-global-graph-parameters}}

Global networks measures at different time points after stroke are
depicted in relation to lesion volume in Fig. 4. Non-linear-mixed
effects modelling revealed a significant positive association between
lesion volume and global connectivity decline in ipsilesional but not
contralesional hemispheres. This effect did not depend on age or sex of
the patient, nor on the side of the lesion. Specifically, larger
declines in ipsilesional median connectivity over time were observed in
patients with larger stroke lesions, while there was no significant
decline in patients with very small lesions. Orthogonally, median
connectivity in stroke hemispheres did not depend on lesion volume in
the acute phase, but a significant negative association was observed at
all three later time points. Similar effects were observed for global
graph parameters. Specifically, larger lesion volumes were associated
with a larger decline in ipsilesional global efficiency, as well as a
larger increase in ipsilesional modularity. Ipsilesional measures of
network topology were associated with lesion volume in the subacute and
chronic, but not the acute phase. There was no evidence of a
relationship between size of the infarct and contralesional network
metrics. Statistical details are provided in Tab. 4.

\begin{figure}
\centering
\includegraphics{results-clinical_files/figure-latex/unnamed-chunk-18-1.pdf}
\caption{Fig. 4: Relation between global network measures of stroke and
intact hemispheres, and stroke lesion volume. Line segments represent
cross-sectional predicted means of network measures in the acute (3-5d,
solid), subacute (1m, dotted), and chronic (3m and 12m, dashed) phases
after stroke. d=days, m=months.}
\end{figure}

\begin{Shaded}
\begin{Highlighting}[]
\NormalTok{temp}
\end{Highlighting}
\end{Shaded}

\begin{verbatim}
## TypeError: Attempting to change the setter of an unconfigurable property.
## TypeError: Attempting to change the setter of an unconfigurable property.
\end{verbatim}

\includegraphics[width=7.20in,height=5.03in,keepaspectratio]{results-clinical_files/figure-latex/unnamed-chunk-19-1.png}
Tab. 4: Point estimates, standard errors, and p-values of model
parameters obtained from fitting the exponential model (1) to temporal
profiles of global graph measures. In the joint model, total change Δ is
modelled as a linear function of log lesion volume, with both intercept
and slope allowed to vary between stroke und intact hemispheres.
Standard errors and \(p\)-values result from non-linear mixed-effects
regressions fit either jointly (`joint model') or separately for stroke
and intact hemispheres, using the R package nlme. s.e.=standard error.

In a univariate sensitivity analysis, these effects were stable across
network densities imposed by proportional thresholding of network
matrices (Supplement).

\hypertarget{association-of-network-properties-with-clinical-variables}{%
\subsubsection{Association of network properties with clinical
variables}\label{association-of-network-properties-with-clinical-variables}}

\hypertarget{global-network-measures}{%
\paragraph{Global network measures}\label{global-network-measures}}

Fig. 5 shows clinical outcome parameters in relation to change in global
network metrics. Two-stage regressions revealed a significant
association between decline of ipsilesional median connectivity until
one, three and twelve months after stroke and NIHSS score (p=0.0087),
relative grip strength (p=0.0023), and FM score (p=0.0013) at these time
points. Similarly, loss of global efficiency and gain of global
modularity in stroke hemispheres was associated with higher NIHSS scores
(p\textsubscript{Eff}=0.0195, p\textsubscript{Mod}=0.0013) as well as
lower relative grip strengths (p\textsubscript{Eff}=0.0479,
p\textsubscript{Mod}=0.0087) and lower FM scores
(p\textsubscript{Eff}=0.0303, p\textsubscript{Mod}=0.0017).

\includegraphics{results-clinical_files/figure-latex/unnamed-chunk-21-1.pdf}
Tab. 6 reports statistical details of volume-corrected regressions.
After including lesion size as a nuisance regressor, the associations of
change in global network architecture and NIHSS persisted at a lower
statistical siginificance. The relationship between global modularity
and relative grip strength, and global efficiency and Fugl-Meyer score
failed to maintain statistical significance.

\begin{Shaded}
\begin{Highlighting}[]
\NormalTok{temp}
\end{Highlighting}
\end{Shaded}

\begin{verbatim}
## TypeError: Attempting to change the setter of an unconfigurable property.
## TypeError: Attempting to change the setter of an unconfigurable property.
\end{verbatim}

\includegraphics[width=7.82in,height=2.62in,keepaspectratio]{results-clinical_files/figure-latex/unnamed-chunk-22-1.png}
Tab. 6: Regression coefficients (point estimates, standard errors and
p-values) on the link scale between change in global network measures
and clinical outcome. Log lesion volume is included as a nuisance
predictor. In the case of NIHSS and FM\^{}*\^{}=66-FM scores,
quasi-Poisson regressions with a \(\log\)-link are used; in the case of
relative grip strength a Gaussian regression with identity link is used.
The first three columns represent pooled estimates from joint two-stage
regressions across the subacute and chronic stages. NIHSS = National
Institute of Health Stroke Scale, rGS = relative grips strength,
FM=Fugl-Meyer, s.e.=standard error

Post-hoc tests for associations between decline of connectivity and
clinical outcome at fixed time points revealed consistent effects that
were strongest after three months, but did not, individually, reach
statistical significance (see Fig SX in the supplement for a visual
representation of cross-sectional regressions).

\hypertarget{local-graph-measures}{%
\paragraph{Local graph measures}\label{local-graph-measures}}

Mass-univariate two-stage linear and quasi-Poisson regressions
identified associations between clinical outcome and change in local
connectivity (strength) in a total of 14 brain areas. Higher residual
NIHSS scores were most strongly associated with connectivity decline in
pre-/paracental, inferior frontal, middle-/superior temporal gyri, as
well as the thalamus, posterior cingulum and visual areas (cuneus,
pericalcarine gyrus). Statistical details including lesion volume
corrected regression results are provided in the Supplement.


\end{document}
